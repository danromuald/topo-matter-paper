\section{\label{sec:intro}Introduction}
The Nobel Prize in Physics 2016 was sahared among David J.Thouless (half of the prize), F. Duncan Haldane ($\frac{1}{4}$ of the prize), and J. Michael Korstelitz ($\frac{1}{4}$ of the prize), for \textbf{\textit{"theoretical discoveries of topological phase transitions and topological phases of matter"}}. The Nobel prize rewarded Thouless and Haldane for their contributions in understanding topological phases of matter \cite{thouless_quantized_1982}, and rewarded Korstelitz and Thouless for their contributions to understanding topological phase transitions \cite{kosterlitz_ordering_1973}. The idea could be simplified as: looking at an electron moving in a solid and exploring its wave function using topological concepts, one can explain a number of experiments. Kosterlit and Thouless introduced a new order parameter for phase transition based on principles of topology. They introduced a new definition of long range order based on golbal properties of a low-dimensional solid, as opposed to interactions between two points (spin-spin interactions for example) in the solid \cite{kosterlitz_ordering_1973}. This long range, "topological order" exists in 2D solids and neutral superfluids at a finite temperature, and is due to the proliferation of the unbinding of vortex-antivortex pairs also referred to as \textit{topological defects} in the 2D solid. 
% \textcolor{red}{In the case of a solid, the disappearance of topological long-range order is associated with a transition from an elastic to a fluid response to a small external shear stress, while for a neutral superfluid it is associated with the instability of persistent currents.} 
The idea of having a phase transition based on topological defects is now studied in many low-dimensional systems such as superfluid $^4He$, 2D Bose-Einstein condensates, superconductors.
\begin{description}
    \item[Order parameters vs topological invariants] what 
\end{description} 